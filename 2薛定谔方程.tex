\chapter{薛定谔方程}
\begin{center}
    \textcolor[RGB]{255, 0, 0}{\faHeart}揭示是量子世界需要一种深邃的直觉,而不仅仅是数学上的技巧。\textcolor[RGB]{255, 0, 0}{\faHeart}
\end{center}
\rightline{里奇·弗里曼(Rich Freeman)}
\vspace{-5pt}
\begin{center}
    \pgfornament[width=0.36\linewidth,color=lsp]{88}
\end{center}

\section{波动方程}
类比与牛顿力学中的$F=ma$,它是怎么来的?如何理解?

$\text{德布罗意:物质是波!} \longrightarrow \text{德拜:找波的方程。} \longrightarrow \text{薛定谔:我找到了!}$

牛顿:自由粒子做匀速直线运动,其波为平面波。

类比:物质是波。

$$\text{平面波的波函数:}\Longrightarrow \text{微观粒子:}$$
$$
\boldsymbol{\phi }(\boldsymbol{x,t})=\left\{ \begin{array}{l}
  e^{i\left( k\boldsymbol{x}-\omega \boldsymbol{t} \right)}\\
  \sin \left( k\boldsymbol{x}-\omega \boldsymbol{t} \right)\\
  \cos \left( k\boldsymbol{x}-\omega \boldsymbol{t} \right)\\
\end{array} \right. \xrightarrow[E=\hbar \omega]{\boldsymbol{p}=\hbar \boldsymbol{\nu }}e^{\frac{i}{\hbar }\left( \boldsymbol{px}-Et \right)}=\boldsymbol{\phi} \left( \boldsymbol{x,t} \right) 
$$
首先看平面波满足的方程:

随时间变化:
\begin{equation}\label{inter}
    \frac{\partial}{\partial t}e^{\frac{i}{\hbar}(\boldsymbol{px}-Et)}=-\frac{i}{\hbar}Ee^{\frac{i}{\hbar}(\boldsymbol{px}-Et)}
\end{equation}

随空间变化:
\begin{equation}\label{inter}
    \frac{\partial}{\partial \boldsymbol{x}}e^{\frac{i}{\hbar}(\boldsymbol{px}-Et)}=\frac{i}{\hbar}\boldsymbol{p}e^{\frac{i}{\hbar}(\boldsymbol{px}-Et)}
\end{equation}

在上面两式中:$\frac{\partial}{\partial t}$ 和 $\frac{\partial}{\partial \boldsymbol{x}}$ 为算符,$-\frac{i}{\hbar}E$ 和 $\frac{i}{\hbar}\boldsymbol{p}$ 为力学量,可以看出,两组微分算符和力学量之间有某种对应关系。

\textcolor{blue}{力学量是算符!(小海:力学量是矩阵)}

自由粒子满足的方程:$E=\frac{p^2}{2m}$ , $i\hbar \frac{\partial}{\partial t}\psi \left( \boldsymbol{x,t} \right) =-\frac{\hbar ^2}{2m}\frac{d^2}{dx^2}\psi \left( \boldsymbol{x,t} \right) $

\textcolor{gray}{算子作用到函数上才有意义}

薛定谔猜:不自由的粒子(带外力):$E=\frac{\left| \boldsymbol{\vec{p}} \right|^2}{2m}+\boldsymbol{V}\left( \boldsymbol{\vec{x},t} \right) $ , $E\rightarrow i\hbar \frac{\partial}{\partial t}$ , $\boldsymbol{\vec{p}}\rightarrow -i\hbar \nabla $ (\textcolor{red}{力学量是算符!})

代换得:$i\hbar \frac{\partial}{\partial t}\psi \left( \boldsymbol{\vec{x},t} \right) =\left[ -\frac{\hbar ^2}{2m}\nabla ^2+\boldsymbol{V}\left( \boldsymbol{x,t} \right) \right] \psi \left( \boldsymbol{\vec{x},t} \right) $

但问题是:这种猜想与结果是否正确呢?

检验:用S-eq算一个试验能测的,氢原子线系。(势能带入库仑势,解出能量)

其它情况:
\begin{enumerate}
\item 相对论:

$$E=\sqrt{p^2c^2+m_0^2c^4} \xrightarrow{\text{根号不解析}} E^2=p^2c^2+m_0^2c^4 \Longrightarrow -\frac{\hbar ^2}{c^2}\frac{\partial ^2}{\partial t^2}\psi =\left( -\hbar ^2\nabla ^2+m_0^2c^2 \right)\psi \text{——克莱因}·\text{高登方程}$$

K-G方程没有开方,平方解含有正负能(负能困难-负几率困难),Dirac方程做了开方,但不是真正意义上的开方,也含有正负能。
\item 多粒子系统

$E=\sum_{i=1}^n{\frac{\left| \boldsymbol{p}_{\boldsymbol{i}} \right|^2}{2m_i}}+V\left( \vec{x}_1,\vec{x}_2,...,\vec{x}_n \right) $ (这个V比较复杂,动能相加,但是势能不能。)

$E\rightarrow i\hbar \frac{\partial}{\partial t},\boldsymbol{\vec{p}}_i\rightarrow -i\hbar \nabla _i$

$i\hbar \frac{\partial}{\partial t}\psi \left( \vec{x}_1,\vec{x}_2,...,\vec{x}_n,t \right) =\left[ \sum_{i=1}^n{-\frac{\hbar ^2}{2m_{i}^{2}}\nabla _{i}^{2}}+V\left( \vec{x}_1,\vec{x}_2,...,\vec{x}_n \right) \right] $

\textcolor{blue}{注:平面波为什么不用$sin(kx-\omega t),cos(kx-\omega t)$ ?\\ 其实最后也能得到一个方程,但是用氢原子检验结果对不上。}
\end{enumerate}




\section{波函数}


$$
\text{物质}\left\{ \begin{matrix}
    \text{粒子}& \rightarrow& \text{玻尔}& \rightarrow& \text{海森堡}\\
    \text{波}& \rightarrow& \text{德布罗意}& \rightarrow& \text{薛定谔}\\
\end{matrix} \right. 
$$

波粒二象性:粒子不是波组成的;波不是粒子的集体运动。

(1)粒子不是波组成的:波包的行为像粒子,但是若粒子是波包$\Rightarrow$真空色散

自由粒子的动量完全确定(匀速)$\Rightarrow$则位置完全不确定(全空间弥散)。

若将波包定域化?

回顾牛顿:光通过介质(棱镜)会色散(速度不同),但是真空中不会(c不变)。\textcolor{blue}{物质波在真空中会色散,速度不同。(why?)}

(2)波不是粒子的集体运动(由单光子干涉试验验证)

(3)玻恩的几率解释(波函数是什么?)单粒子的行为是几率的。

%阅后即焚
\begin{ascolorbox5}[]{阅后即焚}[black!50!white][coltext=cyan!40!white]
    几率解释是物理问题吗?判据:能否被实验验证。………………
\end{ascolorbox5}

\subsection{波函数是什么——几率振幅}
薛定谔方程:

$$i\hbar\frac{\partial}{\partial t}=(-\frac{\hbar^2}{2m}\nabla^2+V)\psi
$$

$\psi$一定是复数,即非几率,若$\psi$ 为实数,则只能为0,因为$i\hbar$ 为纯虚数。

类比光学:用振幅解释干涉

$$\left|\text{振幅}\right|^2=\text{强度} \longrightarrow \left|\psi \right|^2= \text{几率}$$

振幅不可观测,几率可观测。(量子力学也有干涉)

\textcolor{blue}{结论}:$\psi$是几率振幅,$\left|\psi\right|^2$是几率密度。


\subsection{干涉}
叠加的是几率振幅,不是几率。

$$\psi=\psi_1+\psi_2 \text{振幅叠加}$$
$$|\psi|^2=|\psi_1|^2+|\psi_2|^2+\psi_2^*\psi_1+\psi_1^*\psi_2 \text{几率叠加(后两项为干涉项)}$$

\subsection{线性叠加原理}
\textcolor{blue}{结论}:几率振幅是线性叠加。

%随阅随焚
\begin{ascolorbox5}[]{阅后即焚}[black!50!white][coltext=cyan!40!white]
    \begin{enumerate}
    \item S-eq是对的——实验验证
    \item S-eq 是线性方程
    \item 线性方程解可以线性叠加
    \item S-eq解线性叠加
    \item 物理态是S-eq的解
    \item 物理态是线性叠加的
    \end{enumerate}
    因此,试验所说的$\psi$(几率增幅)是线性叠加!
    
    许多书类比光学,光线性叠加$\rightarrow \psi$线性叠加,这种做法不"安全!"

    (经典)光学是线性的——非常偶然,这是因为Maxwell-eq是线性是偶然的。

    光对应电磁作用,是线性的,其余三种作用是非线性的,不可以线性叠加。

    而量子力学与经典电磁理论是线性这件事无关!

    补充:经典电磁学是线性的——光子之间无相互作用,而在量子电磁学中,它们之间存在高阶量子效应(小量)。

\end{ascolorbox5}

\subsection{波函数的归一化问题}
    对于方程$$ i\hbar\frac{\partial}{\partial t} = H \Psi $$
    其为一个其次方程,由实验说明。$\Psi$ 和 $c\Psi$ 都是方程的解 $\Rightarrow$ 选择哪一个呢?

    因此,附加一个要求"归一化条件",但是为什么要归一化?
$$
\left\{ \begin{matrix}
    \text{相信几率解释}& \Rightarrow& \text{几率必须归一}\\
    \text{不信几率解释}& \Rightarrow& \text{那你也可以做一个归"n"化}
\end{matrix}\right.
 $$
$\int_{\Omega}{|\psi| ^2 d\Omega}=1$  \, \,\, $\frac{\int{\psi ^* F\psi \,dx}}{\int{\psi ^* \psi \,dx}}$ \, ,易得归一化常数为$c=\frac{1}{\sqrt{\int_{\Omega}{|\psi|^2 dV}}}$。

\subsection{相因子}
薛定谔方程给出的$\psi$是复数,而归一化条件为实条件,$\psi \Rightarrow c\psi$,c为复数。

结论:归一化条件不能完全确定c,只能确定c的模。
 
?????????????????????后续再做补充。



\subsection{王涛问题}

后续补充













\section{几率(定域)守恒与几率流密度}

牛顿(力学的顶梁柱):粒子数不变$\Rightarrow$ 自由度数不变

$F_x' = m\ddot{x},F_y' = m\ddot{y},F_z' = m\ddot{z},\cdots,$方程数 = 自由度数。

求解过程中方程数不变 $\Rightarrow$ 自由度数不变 $\Rightarrow$ 粒子数不变。

\textcolor{green}{力学中粒子数不变,意味着粒子不能凭空产生或消失}

力学不能解决粒子的产生消失问题$\Rightarrow$若一个过程伴随着粒子的产生或消失问题,则不能用力学描述。

\subsection{经典物理中如何处理产生、消失问题?}

如开灯关灯——光的产生消失(电磁学)。

为什么力学不能描述产生与消失?自由度数不变(方程数不变)。

如果自由度数无穷大?此时无论是产生还是消失,自由度数仍为无穷。

有限大自由度系统中粒子数变化改变自由度,而无穷大没有影响$\Rightarrow$ 用无穷大自由度系统描述产生消失问题。

无穷大自由度系统 $\Rightarrow$ 场(空间中每一个点上都有)

\textcolor{brown}{小结:粒子数的改变可用场来描述。}

例如:光$\Rightarrow$电磁场
$$\text{调光 = 调场强} \left\{ \begin{matrix}
\text{调大:产生光} \\ \text{调小:消失光} 
\end{matrix}\right.$$

\subsection{一个错误}

 $$\text{后续补充}$$




\subsection{定域守恒与流(量子力学中是定域守恒)}

定域守恒:存在流(如大变活人,存在一个“人流———途径”)

量子力学中的流:$$ \vec{j} = -\frac{i \hbar}{2m} (\psi ^* \nabla \psi - \psi \nabla \psi ^*) \xlongequal{\vec{p} = -i\hbar\nabla} \frac{1}{2m}(\psi ^*\vec{p} \psi - \psi \vec{p} \psi ^*) ,\text{\textcolor{red}{核心};} \frac{\vec{p}}{m} = \vec{v} $$

牛顿:单粒子流就是速度

\textcolor{red}{小结:(量子)力学中粒子数定域守恒,存在流}


\section{量子力学的经典极限}
目前认为量子力学是精确的,经典力学是近似的。

\textcolor{blue}{量子$\xlongrightarrow{h \rightarrow 0}$经典}

例:(测不准关系)

\textcolor{red}{需要补充,且请了解泊松括号与对易括号,正则量子化条件。}



\section{定态方程}

$$i\hbar\frac{\partial\psi}{\partial t} = (-\frac{\hbar^2}{2m}\nabla^2 + V)\psi,(-\frac{\hbar^2}{2m}\nabla^2 + V = H,\text{即哈氏量},V=V(x),闭系)$$

$$\left\{ \begin{matrix}
\text{封闭系:H不含t  ——  H就是能量,封闭系能量守恒。}\\
\text{开系:H可以含t(但非必要,比如定态)}
\end{matrix} \right.$$

从某种意义上,闭系更为基本,开系是无奈之举(闭系解决不了,弄成开系)。

H不含t,$i\hbar \frac{\partial \psi}{\partial t}=H\psi $,可以分离变量得:$\psi(\vec{r},t)=\varphi(\vec{r})f(t) $,从而有:
$$ i \hbar \frac{\partial\varphi(\vec{r})f(t)}{\partial t} = H\varphi(\vec{r})f(t) $$
$$ \frac{i\hbar}{f(t)} \frac{df(t)}{dt} = \frac{1}{\varphi(\vec{r})} [-\frac{\hbar^2}{2m} \nabla^2 +V(\vec{r})]\varphi(\vec{r}) = E $$
$$ \left\{ \begin{matrix}
i\hbar\frac{df(t)}{dt} = Ef(t)  \rightarrow  f(t) = ce^{-\frac{i}{\hbar}Et}\\H\varphi = E\varphi \,\, ,(H=T+V)
\end{matrix} \right\} \varphi(\vec{r},t) = \varphi_E(\vec{r})e^{-\frac{i}{\hbar}Et}$$
对于封闭系而言,核心问题变为$H\varphi _E(\vec{r}) = E\varphi_E(\vec{r}) $

\subsection{玻尔定态(德布罗意的解释过于依赖封闭轨道) }

