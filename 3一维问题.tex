\chapter{一维问题}
\begin{center}
    \textcolor[RGB]{255, 0, 0}{\faHeart}我们只能描述现象,不能透过它看清实质。\textcolor[RGB]{255, 0, 0}{\faHeart}
\end{center}
\rightline{——点波推测}
\vspace{-5pt}
\begin{center}
    \pgfornament[width=0.36\linewidth,color=lsp]{88}
\end{center}

\section{一维定态薛定谔方程、本征方程}

\subsection{一维定态S-eq(一维问题可以普遍推广)}

$$
i\hbar \frac{\partial}{\partial t}\psi(x,t) = [-\frac{\hbar^2}{2m} \frac{\partial^2}{\partial x^2} + V(x)] \psi(x,t)
$$
定态方程:$$
[-\frac{\hbar^2}{2m} \frac{d^2}{d x^2} + V(x)] \, \psi(x) = E \,\psi(x), \,\, \psi\text{为本征函数,E为本征值。}$$
回顾:本征函数,本征值。
$$\left\{ \begin{matrix}
\text{线性代数:有限维矩阵}\\
\text{量子力学:算子(哈氏表述)} \Rightarrow \text{无限维矩阵}

\end{matrix}\right.$$

\begin{table}[htbp]
\centering

%下面一行命令用来控制行高
%可用 \setlength{\tabcolsep}{1pt} 来调整表格的列间距离 (十分推荐) 。
%可用 \renewcommand\arraystretch{1.5} 来调整表格行间距,意思是将每一行的高度变为原来的1.5倍 (十分推荐) 。
%如果表格太大,可以使用 \scalebox{1.5} 来对表格进行缩放,意思是将表格的大小变为原来的1.5倍 (十分推荐),使用的时候需要添加包 \usepackage{graphicx} 。
                        
\renewcommand\arraystretch{1.5}{

\begin{tabular}{|c|c|c|}
\hline
        &有限维矩阵 & 无限维矩阵 \\ \hline
    迹  & 对角元和$\sum_{i}^{n}{\lambda_i}$一定有限  & 对角元和$\sum_{i}^{n}{\lambda_i}$不一定有限 \\ \hline
行列式(特殊情况可对角化) & 本征值之积$\prod_{i}^{n}{\lambda_i}$一定有限 & 本征值之积$\prod_{i}^{n}{\lambda_i} $不一定有限  \\
\hline
\end{tabular}}
\end{table}


\subsubsection{本征问题——对角问题}
矩阵对角化$
\left( \begin{matrix}
    a&      \cdots&     c\\
    \vdots&     \ddots&     \vdots\\
    g&      \cdots&     i\\
\end{matrix} \right) \xrightarrow{\text{等价变换}}\left( \begin{matrix}
    A&      \cdots&     0\\
    \vdots&     \ddots&     \vdots\\
    0&      \cdots&     C\\
\end{matrix} \right) 
$

Q:在什么意义下等价?

线性代数出发点:对于线性方程组

$$
\boldsymbol{Ax}=\boldsymbol{b} \Longleftrightarrow
\boldsymbol{A}=\left[ \begin{matrix}{l}
    a_{11}&     a_{12}&     \cdots&     a_{1n}\\
    a_{21}&     a_{22}&     \cdots&     a_{2n}\\
    \vdots&     \vdots&     \ddots&     \vdots\\
    a_{n1}&     a_{n2}&     \cdots&     a_{nn}\\
\end{matrix} \right] ,\,\,\boldsymbol{x}=\left[ \begin{array}{c}
    x_1\\
    x_2\\
    \vdots\\
    x_n\\
\end{array} \right] ,\,\,\boldsymbol{b}=\left[ \begin{array}{c}
    b_1\\
    b_2\\
    \vdots\\
    b_n\\
\end{array} \right] 
$$
建立增广矩阵
$$
\boldsymbol{\bar{A}}=\left[ \begin{matrix}
    \boldsymbol{A}&     \boldsymbol{b}\\
\end{matrix} \right] =\left[ \begin{matrix}
    a_{11}&     a_{12}&     \cdots&     a_{1n}&     b_1\\
    a_{21}&     a_{22}&     \cdots&     a_{2n}&     b_2\\
    \vdots&     \vdots&     \ddots&     \vdots&     \vdots\\
    a_{n1}&     a_{n2}&     \cdots&     a_{nn}&     b_n\\
\end{matrix} \right] 
\xlongrightarrow[\text{化为如下形式的梯形矩阵}]{\text{若能将增广矩阵}\boldsymbol{\bar{A}}\text{做初等行变换}}
\left[ \begin{matrix}
    1&      &       &       &       \xi _1\\
    &       1&      &       &       \xi _2\\
    &       &       \ddots&     &       \vdots\\
    &       &       &       1&      \xi _n\\
\end{matrix} \right] 
$$
$
\text{则}\xi _1,\xi _2,\cdots \xi _n\text{即为方程组的解}\left\{ \begin{array}{l}
    a_{11}x_1+a_{12}x_2+\cdots +a_{1n}x_n=0\\
    a_{21}x_1+a_{22}x_2+\cdots +a_{2n}x_n=0\\
    \cdots \,\,\cdots \,\,\cdots \,\,\cdots\\
    a_{m1}x_1+a_{m2}x_2+\cdots +a_{mn}x_n=0\\
\end{array} \right. 
$

上述过程中对矩阵进行了等价变换,这种等价是\textcolor{red}{解不变意义下的等价。}

矩阵一定可以化为约当标准型,但不一定可以对角化。

在物理中遇到的问题通常特殊,具有良好的数学性质,数学问题往往逻辑自洽,考虑各种极限条件,太过于“抽象”,没有现实性。

本征矢:$$\text{本征关系}\left\{ \begin{matrix}
\text{线性代数:}(\alpha\,\beta\,\gamma)\left(\begin{matrix}a\\b\\c
\end{matrix}\right)= \left(\begin{matrix}d\\e\\f\end{matrix}\right) \xlongrightarrow{\text{如果成立}} \lambda\left(\begin{matrix}a\\b\\c\end{matrix}\right)\\
\text{量子力学:} \hat{D}\psi \, \xlongrightarrow{\text{如果成立}} \, \Lambda \psi
\end{matrix}\right.
$$

\subsubsection{一维定态薛定谔方程(核心问题)}






\section{简并}








\section{一维定态波函数可表示为实函数}









\section{宇称}








\section{一维束缚态本征函数的节点}







\section{奇异势、无限深势阱}










\section{有限深势阱}
























